\documentclass[10pt]{beamer}

\usecolortheme{seahorse}

% pour utiliser une police à empattements partout
\usefonttheme{serif}

% pour rajouter la numérotation des frames dans les pieds de page
\newcommand*\oldmacro{}%
\let\oldmacro\insertshorttitle%
\renewcommand*\insertshorttitle{%
  \oldmacro\hfill%
  \insertframenumber\,/\,\inserttotalframenumber}

}

\usepackage{graphicx} % allows including images
\usepackage{booktabs} % allows the use of \toprule, \midrule and \bottomrule in tables

\atbeginsection[]
{
\begin{frame}{plan}
\tableofcontents[currentsection]
\end{frame}
}
\input{preambule/special_beamer.tex}

\usepackage{natbib}         % Pour la bibliographie
\usepackage{url}            % Pour citer les adresses web
\usepackage[utf8]{inputenc} % Lui aussi
\usepackage[frenchb]{babel} % Pour la traduction française
\usepackage{numprint}       % Histoire que les chiffres soient bien

\usepackage{amsmath}        % La base pour les maths
\usepackage{mathrsfs}       % Quelques symboles supplémentaires
\usepackage{amssymb}        % encore des symboles.
\usepackage{amsfonts}       % Des fontes, eg pour \mathbb.

\usepackage{pifont}
\usepackage{cancel}
\usepackage{hhline}


\usepackage{graphicx} % inclusion des graphiques

\usepackage{tikz}
\usepackage[framemethod=TikZ]{mdframed}

% Ce fichier contient toutes les macros que vous pouvez avoir envie de définir
% si vous les utilisez plusieurs fois dans le document.

\PassOptionsToPackage{svgnames}{color}

% Un environnement pour bien présenter le code informatique
\newenvironment{code}{%
\begin{mdframed}[linecolor=green,innerrightmargin=30pt,innerleftmargin=30pt,
backgroundcolor=black!5,
skipabove=10pt,skipbelow=10pt,roundcorner=5pt,
splitbottomskip=6pt,splittopskip=12pt]
}{%
\end{mdframed}
}


\newcommand\N{\mathbb{N}}
\newcommand\R{\mathbb{R}}
\newcommand\E{\mathbb{E}}
\newcommand\Pb{\mathbb{P}}
\newcommand\va[1]{\mathbf{#1}}


\newcommand{\argmin}{\mathop{\arg\min}}                     % Arg-min
\newcommand{\st}{\text{s.t.}}                            % ``s.t.''

%%%%% Math\'{e}matiques g\'{e}n\'{e}rales
\newcommand{\defegal}{:=}                                   % D\'{e}finition

\newcommand{\bgfset}[2]{\big\{#1\:\big|\:#2\big.\big\}}     % Ensemble
\newcommand{\Bgfset}[2]{\Big\{#1\:\Big|\:#2\Big.\Big\}}
\newcommand{\defset}[2]{\left\{#1\:\left|\:#2\right.\right\}}

\newcommand{\IFF}{iff~}                                     % if and only if

\newcommand{\bbN}{\mathbb{N}}                               % Entiers naturels
\newcommand{\bbR}{\mathbb{R}}                               % Nombres r\'{e}els
\newcommand{\bbRb}{\overline{\mathbb{R}}}                   % Completion de R

\newcommand{\abs}[1]{\left|#1\right|}                       % Valeur absolue
\newcommand{\norm}[1]{\left\|#1\right\|}                    % Norme
\newcommand{\sqnorm}[1]{\left\|#1\right\|^{2}}              % Norme au carr\'{e}

\newcommand{\dd}{\,\mathrm{d}}                              % d de derivation
\newcommand{\dpp}[2]{\dd #1 (#2)}                           % dP(omega)

\newcommand{\derpar}[2]{\frac{\partial#1}{\partial#2}}      % D\'{e}riv\'{e}e partielle
\newcommand{\dertot}[2]{\frac{\mathrm{d}#1}{\mathrm{d}#2}}  % D\'{e}riv\'{e}e totale

\newcommand{\projop}[1]{\mathrm{proj}_{#1}}                 % Op\'{e}rateur projection
\newcommand{\proj}[2]{\projop{#1}\left(#2\right)}           % Projection
\newcommand{\dist}[2]{\mathrm{dist}_{#1}\left(#2\right)}    % Distance

%%%%% Parenth\`{e}ses, crochets et accolades

\newcommand{\np}[1]{(#1)}                                   % Parenth\`{e}se normal
\newcommand{\bp}[1]{\big(#1\big)}                           % Parenth\`{e}se big
\newcommand{\Bp}[1]{\Big(#1\Big)}                           % Parenth\`{e}se Big
\newcommand{\bgp}[1]{\bigg(#1\bigg)}                        % Parenth\`{e}se bigg
\newcommand{\Bgp}[1]{\Bigg(#1\Bigg)}                        % Parenth\`{e}se Bigg

\newcommand{\nc}[1]{[#1]}                                   % Crochet normal
\newcommand{\bc}[1]{\big[#1\big]}                           % Crochet big
\newcommand{\Bc}[1]{\Big[#1\Big]}                           % Crochet Big
\newcommand{\bgc}[1]{\bigg[#1\bigg]}                        % Crochet bigg
\newcommand{\Bgc}[1]{\Bigg[#1\Bigg]}                        % Crochet Bigg

\newcommand{\na}[1]{\{#1\}}                                 % Accolade normal
\newcommand{\ba}[1]{\big\{#1\big\}}                         % Accolade big
\newcommand{\Ba}[1]{\Big\{#1\Big\}}                         % Accolade Big
\newcommand{\bga}[1]{\bigg\{#1\bigg\}}                      % Accolade bigg
\newcommand{\Bga}[1]{\Bigg\{#1\Bigg\}}                      % Accolade Bigg

%%%%% Versions avec s\'{e}parateur

\newcommand{\nps}[2]{\np{#1\mid#2}}                         % Parenth\`{e}se normal
\newcommand{\bps}[2]{\bp{#1\ \big|\ #2}}                    % Parenth\`{e}se big
\newcommand{\Bps}[2]{\Bp{#1\ \Big|\ #2}}                    % Parenth\`{e}se Big
\newcommand{\bgps}[2]{\bgp{#1\ \bigg|\ #2}}                 % Parenth\`{e}se bigg
\newcommand{\Bgps}[2]{\Bgp{#1\ \Bigg|\ #2}}                 % Parenth\`{e}se Bigg

\newcommand{\ncs}[2]{\nc{#1\mid#2}}                         % Crochet normal
\newcommand{\bcs}[2]{\bc{#1\ \big|\ #2}}                    % Crochet big
\newcommand{\Bcs}[2]{\Bc{#1\ \Big|\ #2}}                    % Crochet Big
\newcommand{\bgcs}[2]{\bgc{#1\ \bigg|\ #2}}                 % Crochet bigg
\newcommand{\Bgcs}[2]{\Bgc{#1\ \Bigg|\ #2}}                 % Crochet Bigg

\newcommand{\nas}[2]{\na{#1\mid#2}}                         % Accolade normal
\newcommand{\bas}[2]{\ba{#1\ \big|\ #2}}                    % Accolade big
\newcommand{\Bas}[2]{\Ba{#1\ \Big|\ #2}}                    % Accolade Big
\newcommand{\bgas}[2]{\bga{#1\ \bigg|\ #2}}                 % Accolade bigg
\newcommand{\Bgas}[2]{\Bga{#1\ \Bigg|\ #2}}                 % Accolade Bigg

%%%%% Raccourcis package maths
\newcommand{\cA}{\mathcal{A}}
\newcommand{\cB}{\mathcal{B}}
\newcommand{\cC}{\mathcal{C}}
\newcommand{\cD}{\mathcal{D}}
\newcommand{\cE}{\mathcal{E}}
\newcommand{\cF}{\mathcal{F}}
\newcommand{\cG}{\mathcal{G}}
\newcommand{\cH}{\mathcal{H}}
\newcommand{\cI}{\mathcal{I}}
\newcommand{\cJ}{\mathcal{J}}
\newcommand{\cK}{\mathcal{K}}
\newcommand{\cL}{\mathcal{L}}
\newcommand{\cM}{\mathcal{M}}
\newcommand{\cN}{\mathcal{N}}
\newcommand{\cO}{\mathcal{O}}
\newcommand{\cP}{\mathcal{P}}
\newcommand{\cQ}{\mathcal{Q}}
\newcommand{\cR}{\mathcal{R}}
\newcommand{\cS}{\mathcal{S}}
\newcommand{\cT}{\mathcal{T}}
\newcommand{\cU}{\mathcal{U}}
\newcommand{\cV}{\mathcal{V}}
\newcommand{\cW}{\mathcal{W}}
\newcommand{\cX}{\mathcal{X}}
\newcommand{\cY}{\mathcal{Y}}
\newcommand{\cZ}{\mathcal{Z}}

\newcommand{\BB}{\mathbb{B}}
\newcommand{\CC}{\mathbb{C}}
\newcommand{\DD}{\mathbb{D}}
\newcommand{\EE}{\mathbb{E}}
\newcommand{\FF}{\mathbb{F}}
\newcommand{\GG}{\mathbb{G}}
\newcommand{\HH}{\mathbb{H}}
\newcommand{\II}{\mathbb{I}}
\newcommand{\JJ}{\mathbb{J}}
\newcommand{\KK}{\mathbb{K}}
\newcommand{\LL}{\mathbb{L}}
\newcommand{\MM}{\mathbb{M}}
\newcommand{\NN}{\mathbb{N}}
\newcommand{\OO}{\mathbb{O}}
\newcommand{\PP}{\mathbb{P}}
\newcommand{\QQ}{\mathbb{Q}}
\newcommand{\RR}{\mathbb{R}}
\newcommand{\TT}{\mathbb{T}}
\newcommand{\UU}{\mathbb{U}}
\newcommand{\VV}{\mathbb{V}}
\newcommand{\WW}{\mathbb{W}}
\newcommand{\XX}{\mathbb{X}}
\newcommand{\YY}{\mathbb{Y}}
\newcommand{\ZZ}{\mathbb{Z}}


\title{Machine Learning with Artelys Knitro}
\date{May 29th, 2020}


\begin{document}

\begin{frame}{Une modélisation en plusieurs étapes}
  Travail adapté de Cosma Shalizi
\end{frame}

\section{Le modèle SIR}

\begin{frame}{Un premier modèle probabiliste}
  \begin{itemize}
    \item En cas d'épidémie, chaque personne peut être dans un des trois états
      \begin{itemize}
        \item \emph{Sain (S)}: non malade, mais peut être contaminé
        \item \emph{Infectieux (I)}: malade, et peut contiminer d'autres personnes
        \item \emph{Removed (R)}: guéri
      \end{itemize}
  \end{itemize}

  \begin{block}{Hypothèses}
    \begin{itemize}
      \item Contagion : un $S$ peut être contaminé quand il rencontre un $I$,
      \item Removal: au bout d'un certain temps, les $I$ se transforment spontanément en $R$
      \item Mixing: la \emph{probabilité} qu'un $S$ rencontre un $I$ dépend du nombre total
        de $S$, $I$ et $R$
    \end{itemize}
  \end{block}
\end{frame}

\begin{frame}{Un premier modèle probabiliste}
  Proposons un premier modèle discret, suivant les hypothèses précédentes
  \begin{itemize}
    \item La population totale est $n$
    \item

  \end{itemize}

\end{frame}

\begin{frame}{Et nous obtenons les équations probabilistes ...}
  Le modèle s'écrit dès-lors :
  \begin{equation}
    \left\{
      \begin{aligned}
        C_{k} &~ \text{binom}(S_k, \frac{\beta}{n}\Delta I) \\
        D_{k} &~ \text{binom}(I_k, \gamma \Delta) \\
        S_{k+1} &= S_k - C_k \\
        I_{k+1} &= I_k + C_k - D_k \\
        R_{k+1} &= R_k + D_k
      \end{aligned}
    \right.
  \end{equation}
\end{frame}

\begin{frame}{Simulation (1)}

\end{frame}

\begin{frame}{Simulation (2)}

\end{frame}

\begin{frame}{Simulation (3)}

\end{frame}

\begin{frame}{Et si nous passions à la limite ?}
  Que se passe-t-il si la population devient très grande ($n \to \infty$) et
  le pas de temps très petit ($\Delta \to 0$) ?
  \begin{itemize}
    \item On a:
    \[
      \dfrac{S_{k+1} - S_k}{\Delta} =
      \quad
      \dfrac{I_{k+1} - I_k}{\Delta} =

    \]
  \item En passant à l'espérance:

  \end{itemize}

\end{frame}

\begin{frame}{Le modèle SIR}
  Le modèle SIR correspond aux équations différentielles suivantes :
  \begin{equation}
    \left\{
      \begin{aligned}
        & \dot{S} = - \frac{\beta}{n} S I \\
        & \dot{I} = \frac{\beta}{n} I S - \gamma I \\
        & \dot{R} = \gamma I
      \end{aligned}
    \right.
  \end{equation}
  Ces équations sont déterministes !
\end{frame}

\begin{frame}{Qu'est-ce que $R_0$?}
  En normalisant les équations du modèle SIR, nous observons que le seul
  paramètre restant est le ratio $\frac{\beta}{\gamma}$.

  \begin{block}{Nombre de reproduction basique $R_0$}
   On définit $R_0$ comme le nombre moyen de nouvelles infections si on
   ajoute un agent infectieux dans une population non atteinte
  \end{block}

  Nous avons trois régimes :
  \begin{itemize}
    \item $R_0 < 1$: Sous-critique
    \item $R_0 > 1$: Super-critique
    \item $R_0 = 1$: Critique
  \end{itemize}
\end{frame}

\begin{frame}{Identification de $R_0$}
  Dans le modèle SIR, on identifie :
  \begin{equation}
    R_0 = \frac{\beta}{\gamma}
  \end{equation}
  On suppose qu'initialement: $S(0) \approx n$ Alors
  \begin{equation}
    \begin{aligned}
      \dot{I} &= \frac{\beta}{n} S I - \gamma I  \\
              &= (\beta \frac{S}{n} - \gamma) I \\
              &\approx (\beta  - \gamma) I \\
    \end{aligned}
  \end{equation}
  d'où, au début de l'épidémie : $I(t) \approx I_0 e^{(\beta - gamma)t}$

\end{frame}

\begin{frame}{Les limites du modèle SIR}
  Le modèle SIR comporte seulement trois états ! Des variantes existent :
  \begin{itemize}
    \item Etat exposé $E$ des gens ont été exposés à la maladie mais ne sont pas
      encore dans l'état $I$
    \item Etat asymptomatique
  \end{itemize}
\end{frame}

\begin{frame}{Un modèle régionalisé}
  Une extension usuelle du modèle SIR est de prendre en compte les échanges
  entre plusieurs régions $i = 1, \cdots, N$
  \begin{equation}
    \dot{I_i} = \frac{\beta}{n} S_i I_i - \gamma I_i - \mu_{ij} I_i +
    \mu_{ji} I_j
  \end{equation}
  où $\mu_{ij}$ est la proportion de personnes voyageant de la région $i$ à $j$
\end{frame}

\section{Un modèle plus détaillé}

\begin{frame}{Un modèle détaillé : }
  Nous avons vu précédemment le modèle SIR \\
  Analysons maintenant comment pouvons nous adopter ce modèle sur un graphe

\end{frame}

\end{document}
